\documentclass[11pt]{article}
\usepackage[utf8]{inputenc}
\usepackage[a4paper, margin=60pt]{geometry}
\usepackage{amsmath,amsfonts,amsthm}
\usepackage{graphicx}
\usepackage[svgnames]{xcolor}
\usepackage[colorlinks=true, allcolors=Indigo]{hyperref}
\usepackage{textgreek}

\renewcommand\familydefault{\sfdefault} 
\usepackage{setspace} % sets line spacing
\onehalfspacing
\frenchspacing % no extra space after periods
\pagestyle{empty} % no page numbers
\setlength{\parindent}{0pt} % no indent in new paragraphs

\newcommand{\linesep}{\vspace*{1em}}

\newcommand{\MyTitle}[1]{
    {\fontfamily{cmbr} \fontsize{36}{48} \selectfont \color{Indigo} #1} \par
    \vspace*{1em}
}

\newcommand{\MySection}[1]{
    \vspace*{2em}
    {\fontfamily{cmbr} \large \color{Indigo} \uppercase{#1} \hrulefill} \par
    \vspace*{1em}
}

\newcommand{\MySubSec}[1]{
    \vspace*{0.25em}
    {\fontfamily{cmbr} \color{Indigo} \selectfont #1} \par
    \vspace*{0.25em}
}

\newcommand{\InfoItem}[2]{
    \textbf{#1}: #2 \par
}

\newcommand{\EduItem}[6]{
    \textbf{#1}, #2, #3 \hfill \textbf{#4} \par
    {\small Thesis: #5 \par
    Supervisor: #6} \par
}

\newcommand{\PubItem}[3]{
    \textbf{#1} \par
    {\small #2 \par
    #3} \par
}

\newcommand{\GrantItem}[2]{
    \textbf{#1}: #2 \par
}

\newcommand{\EventItem}[5]{
    \textbf{#1}, #2, #3 (#4) \par
    {\small #5} \par
}

\newcommand{\ActivItem}[3]{
    {\small
    \quad \textbf{#1}, #2 \par
    \quad #3} \par
}

\newcommand{\SkillItem}[2]{
    {\small
    \quad \textbf{#1}, #2 \par
    }
}

\begin{document}

\MyTitle{Tadeu Tassis}

%%%%%%%%%%%%%%%%%%%%%%%%%%%%%%%%%%%%%%%%%%%%%%%%%%%%%%%%%%%
\MySection{Personal information}
%%%%%%%%%%%%%%%%%%%%%%%%%%%%%%%%%%%%%%%%%%%%%%%%%%%%%%%%%%%

\InfoItem{Citizenship}{Brazilian}
\InfoItem{Webpage}{\url{https://tadeutassis.github.io}}
\InfoItem{Contact}{tadeutassis@gmail.com}

%%%%%%%%%%%%%%%%%%%%%%%%%%%%%%%%%%%%%%%%%%%%%%%%%%%%%%%%%%%
\MySection{Education}
%%%%%%%%%%%%%%%%%%%%%%%%%%%%%%%%%%%%%%%%%%%%%%%%%%%%%%%%%%%

\EduItem{PhD in Physics}
{Federal University of ABC (UFABC)}
{Brazil}
{2019--Present}
{Trapped ions beyond low intensity regimes}
{Prof. Fernando L. Semião}

\linesep

\EduItem{MSc in Physics}
{Federal University of Espírito Santo (UFES)}
{Brazil}
{2017--2019}
{Topological solitons in scalar field theories in (1+1)-dimensions}
{Prof. Gabriel Luchini}

\linesep

\EduItem{BSc in Physics}
{UFES}
{Brazil}
{2013--2017}
{The Fermi-Pasta-Ulam-Tsingou problem (in Portuguese)}
{Prof. Gabriel Luchini}

%%%%%%%%%%%%%%%%%%%%%%%%%%%%%%%%%%%%%%%%%%%%%%%%%%%%%%%%%%%
\MySection{Publication list}
%%%%%%%%%%%%%%%%%%%%%%%%%%%%%%%%%%%%%%%%%%%%%%%%%%%%%%%%%%%

\PubItem{Trapped ions beyond carrier and sideband interactions}
{T. Tassis and F. L. Semião}
{Physical Review A 107, 042605 (2023)}

\linesep

\PubItem{Collective coordinates for the hybrid model}
{C. F. S. Pereira, E. S. Costa Filho, and T. Tassis}
{International Journal of Modern Physics A 38, 2350006 (2023)}

\linesep

\PubItem{Some novel considerations about the collective coordinates approximation for the scattering of \textphi \textsuperscript{4} kinks}
{C. F. S. Pereira, G. Luchini, T. Tassis, and C. P. Constantinidis}
{Journal of Physics A: Mathematical and Theoretical 54, 075701 (2021)}

\linesep

\PubItem{BPS states for scalar field theories based on g\textsubscript{2} and su(4) algebras}
{G. Luchini and T. Tassis}
{Journal of High Energy Physics 2020, 11 (2020)}

%%%%%%%%%%%%%%%%%%%%%%%%%%%%%%%%%%%%%%%%%%%%%%%%%%%%%%%%%%%
\MySection{Fellowships}
%%%%%%%%%%%%%%%%%%%%%%%%%%%%%%%%%%%%%%%%%%%%%%%%%%%%%%%%%%%

\GrantItem{2021--Present}{PhD Scholarship CAPES}

\GrantItem{2019--2021}{PhD Scholarship from UFABC}

\GrantItem{2017--2019}{MSc Scholarship from CAPES}

\GrantItem{2016--2016}{Undergrad Research Scholarship from UFES}

\GrantItem{2015--2016}{Undergrad Research Scholarship from UFES}

%%%%%%%%%%%%%%%%%%%%%%%%%%%%%%%%%%%%%%%%%%%%%%%%%%%%%%%%%%%
\MySection{Events}
%%%%%%%%%%%%%%%%%%%%%%%%%%%%%%%%%%%%%%%%%%%%%%%%%%%%%%%%%%%

\EventItem{XLI Paulo Leal Ferreira Physics Conference}
{IFT-Unesp}
{Brazil}
{2018}
{\quad Poster: \emph{Scattering of an electron by a Dirac monopole}}

\linesep

\EventItem{II School on Theoretical High Energy Physics}
{IFSC-USP}
{Brazil}
{2018}

\linesep

\EventItem{School on Theoretical High Energy Physics}
{IFSC-USP}
{Brazil}
{2016}

\linesep

\EventItem{Short Course on: Solitons in Classical Field Theories}
{IFSC-USP}
{Brazil}
{2016}

\linesep

\EventItem{XXVI Winter Physics School}
{UFMG}
{Brazil}
{2015}

%%%%%%%%%%%%%%%%%%%%%%%%%%%%%%%%%%%%%%%%%%%%%%%%%%%%%%%%%%%
\MySection{Activities}
%%%%%%%%%%%%%%%%%%%%%%%%%%%%%%%%%%%%%%%%%%%%%%%%%%%%%%%%%%%

\MySubSec{Teaching assistant}

\ActivItem{Classical mechanics II}
{UFABC, Brazil (2022)}
{Lecturer: Prof. Fernando L. Semião}

\ActivItem{Thermal phenomena}
{UFABC, Brazil (2021)}
{Lecturer: Prof. Roberto M. Serra}

\linesep

\MySubSec{Examining commitees}

\ActivItem{BSc thesis defence of João Vitor Bastos Del Piero}
{UFES, Brazil (2019)}

%%%%%%%%%%%%%%%%%%%%%%%%%%%%%%%%%%%%%%%%%%%%%%%%%%%%%%%%%%%
\MySection{Skills}
%%%%%%%%%%%%%%%%%%%%%%%%%%%%%%%%%%%%%%%%%%%%%%%%%%%%%%%%%%%

\begin{minipage}[t]{0.33\linewidth}
    \MySubSec{Languages}

    \SkillItem{Portuguese}{native}

    \SkillItem{English}{advanced}

    \SkillItem{Spanish}{basic}
\end{minipage}
\begin{minipage}[t]{0.66\linewidth}
    \MySubSec{Programming Languages}

    \SkillItem{Python}
    {experience with NumPy, SciPy, Matplotlib, and QuTiP}

    \SkillItem{Julia}
    {experience with QuantumOptics.jl}

    \SkillItem{C/C++}
    {experience implementing basic numerical methods (e.g., RK4)}

    \MySubSec{Software}

    \quad {\small \textbf{Linux}}

    \quad {\small \textbf{\LaTeX}}
\end{minipage}
\end{document}